\documentclass{report}
\usepackage{graphicx}
\graphicspath{{./images/}}
\begin{document}
	\chapter{Introduction}
		\section{Document Purpose}
		\section{Product Scope \& Overview}
  			1. Product Scope
			NGO Aid Operations Management System (NGO MANAGEMENT) is a software solution designed specifically for non-governmental organizations 			(NGOs). This system was developed to help NGOs effectively manage aid operations, optimize resources and track aid activities.

			The main components of NGO MANAGEMENT are:

			Project Management: Includes project management such as project creation, resource planning, and monitoring project progress.

			Donation and Fund Management: Fundraising involves donor management and ensuring that funds are provided effectively.

			Stock Management: Includes storing incoming materials, inventory management and monitoring stock transactions.

			Reporting and Analysis: Includes performing operational analysis, monitoring performance and supporting decision-making processes.

			2. Overview
			NGO MANAGEMENT is a comprehensive solution to optimize the planning, programming and monitoring of NGOs' aid operations. This system 			can be specially made for use by NGOs with its user-friendly interface and modular structure. It is designed to increase the amount 			of NGOs and use resources more effectively.
			
			NGO MANAGEMENT offers a number of features to make it easier to manage complex aid operations. The project management module allows 			NGOs to plan their projects, provide resources and monitor progress. The donation and fund management module automates fundraising 			processes and ensures the efficient generation of donations. The stock management module ensures that incoming materials are stored 			and inventory is stored effectively. The reporting and analysis module helps to visualize the organization of NGOs, monitor their 			performance and take resolved decisions.
			
			NGO MANAGEMENT is designed to help NGOs conduct relief operations more effectively and meet the needs of communities. We aim to 			increase the use of this system by NGOs by using the power of technology.
		\section{Intended Audience \& Document Overview}
		\section{Document Conventions}
		\section{References \& Acknowledgements}
	\chapter{Overall Description}
		\section{Product Overview}
		\section{Product Functionality}
		\section{Design \& Implementation Constraints}
		\section{Assumptions and Dependencies}
	\chapter{Specific Requirements}
		\section{External Interface Requirements}
		\section{Contex Model}
		\section{Functional Requirements}
		\section{Use Case Model}
	\chapter{Other Non Functional Requirements}
		\section{Performance Requirements}
		\section{Safety, Reliability Requirement}
		\section{Security Requirements}
		\section{Software Quality Attributes}
	\chapter{Other Requirements}
		\section{Usage Requirements}
		\section{Efficiency Requirements}
		\section{Resources Requirements}
	\chapter{Requirements Rationale}
\end{document}
